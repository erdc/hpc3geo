\documentclass[12pt,dvips,letterpaper]{article}
\usepackage{epsfig,amsmath,amsfonts}
\begin{document}
%differential operators
\newcommand{\grad}{\nabla}
\newcommand{\deld}{\nabla \cdot}
\newcommand{\lap}{\Delta}
%boldface in math mode
\newcommand{\bm}[1]{\mbox{{\boldmath ${#1}$}}}
% vectors and tensors
\renewcommand{\vec}[1]{{\bf #1}}
\newcommand{\gvec}[1]{\mbox{{\boldmath ${#1}$}}}
\newcommand{\ten}[1]{\bar{\bm{#1}}}
%derivatives
\newcommand{\od}[2]{\frac{d {#1}}{d {#2}}}
\newcommand{\ods}[2]{\frac{d^2{#1}}{d {{#2}^2}}}
\newcommand{\pd}[2]{\frac{\partial {#1}}{\partial {#2}}}
\newcommand{\pds}[2]{\frac{\partial^2{#1}}{\partial {{#2}^2}}}
\newcommand{\pdsm}[3]{\frac{\partial^2{#1}}{\partial {#2}\,\partial {#3}}}
%funtional analysis
\newcommand{\abs}[1]{\left| #1 \right|}
\newcommand{\norm}[1]{\left\| #1 \right\|}
\newcommand{\iprod}[2]{\left( #1, #2 \right)}
\newcommand{\dprod}[2]{\left\langle #1, #2 \right\rangle}
%real numbers
\newcommand{\field}[1]{\mathbb{#1}}
\newcommand{\R}{\field{R}}
%funciton spaces
\newcommand{\M}{\mathcal{M}}
%delimiters
\newcommand{\pl}{\left(}
\newcommand{\pr}{\right)}
\newcommand{\sbl}{\left[}
\newcommand{\sbr}{\right]}
\newcommand{\dbl}{\left[\hspace{-0.05cm}\left[}
\newcommand{\dbr}{\right]\hspace{-0.05cm}\right]}
\newcommand{\cbl}{\left\{ }
\newcommand{\cbr}{\right\} }
\newcommand{\eqn}[1]{equation \ref {eq:#1}} 
\newcommand{\Eqn}[1]{Equation \ref {eq:#1}} 
\newcommand{\eqnst}[2]{equations \ref{eq:#1} and \ref{eq:#2}} 
\newcommand{\Eqnst}[2]{Equations \ref{eq:#1} and \ref{eq:#2}} 
\newcommand{\eqns}[2]{equations \ref{eq:#1}--\ref{eq:#2}} 
\newcommand{\Eqns}[2]{Equations \ref{eq:#1}--\ref{eq:#2}}
\newcommand{\msection}[1]{ \vspace{.2in} {\noindent \bf #1}.}
\renewcommand{\for}{\mbox{for}\quad}
%\newcommand{\for}{\mbox{for}\quad}
\newcommand{\argmin}{\mbox{argmin}}
\newcommand{\argmax}{\mbox{argmax}}
\newcommand{\fig}[1]{figure \ref{fig:#1}} 
\newcommand{\Fig}[1]{Figure \ref{fig:#1}} 
\newcommand{\figst}[2]{figures \ref {fig:#1} and \ref {fig:#2}} 
\newcommand{\Figst}[2]{Figures \ref {fig:#1} and \ref {fig:#2}} 
\newcommand{\figs}[2]{figures \ref{fig:#1}--\ref{fig:#2}} 
\newcommand{\Figs}[2]{Figures \ref{fig:#1}--\ref{fig:#2}}
\newcommand{\tab}[1]{table \ref {tab:#1}} 
\newcommand{\Tab}[1]{Table \ref {tab:#1}} 
\newcommand{\tabst}[2]{tables \ref {tab:#1} and \ref {tab:#2}} 
\newcommand{\Tabst}[2]{Tables \ref {tab:#1} and \ref {tab:#2}} 
\newcommand{\tabs}[2]{tables \ref{tab:#1}--\ref{tab:#2}} 
\newcommand{\Tabs}[2]{Tables \ref{tab:#1}--\ref{tab:#2}}
\newtheorem{theorem}{Theorem}
\newenvironment{neqnarray}[1]{\begin{minipage}[t]{6.5in}  \begin{minipage}[b]{1.0in} #1 \end{minipage}  \begin{minipage}[b]{5.5in}\begin{eqnarray}}{\end{eqnarray}\end{minipage}\end{minipage}}
\newcommand{\bneqnarray}[2]{\\ \\ \fbox{\begin{neqnarray}{#1} #2 \end{neqnarray}}\\ \\ \noindent} 

\newcommand{\R}{\mathbb{R}}
\newcommand{\framedEquation}[1]{\fbox{$ \begin{array}{rcl} #1 \end{array} $}}
\newcommand{\framedArray}[1]{\fbox{$ \begin{array}{rcl} #1 \end{array} $}}

\begin{center}
  Model Formulations for Flow and Transport in Porous Media
\end{center}
\begin{center} 
  C. E. Kees \\
  Center for Research in Scientific Computation \\
  Department of Mathematics \\
  North Carolina State University \\
  Raleigh, NC 27695-8205 \\
  (chris\_kees@ncsu.edu) \\
\end{center}
\begin{abstract}
  I present brief model formulations for flow in porous media,
  focusing on the general structure rather than physical details.
  I put the main model equations in boxes and the solution is
  always labeled $u$. Boundary conditions can be a mixture of
  Dirichlet, Neumann, and Robin conditions and may depend on time.
  Nonlinear source terms can be added to any of the models to
  represent injection or extraction wells. To simplify the notation, I
  will assume that none of the coefficients vary in space, but in most
  cases the coefficients vary significantly.
\end{abstract}
\begin{center}
  Updated: October 1, 2003 \\
\end{center}

\section{Single Phase Flow}

The porous medium is saturated by a single fluid phase, usually water.

\subsection{Incompressible Case}
Assuming the fluid is incompressible, the model is
\begin{equation}
\framedEquation{ - \deld  \ten{a} \grad u  = 0 }
\end{equation}
where $\ten{a}$ is the symmetric positive definite {\em hydraulic conductivity} and
\begin{equation*}
u = \frac{p - p_0}{\rho_0 \| g\|} - \frac{\vec g \cdot \vec x}{\| g\|}
\end{equation*}
is the {\em total head} where $p$ is pressure, $p_0$ is a reference
pressure, $\rho_0$ is the density of water at the reference pressure, and $\vec g$ is the
gravitational acceleration.
\subsection{Compressible Case}
We write the density of water in the porous medium as $\rho \omega$
where $\omega$ is the {\em porosity} or fraction of the porous media
occupied by water. Density and porosity are parameterized as
\begin{eqnarray*}
\rho(\psi) &=& \rho_0 e^{c_0 \psi} \\
\omega(\psi) &=& \omega_0 e^{c_R \psi} \\
\end{eqnarray*}
where $\omega_0$ is the porosity at $p_0$, $c_0$ and $c_R$ are
fluid and media compressibilities, and 
\begin{equation*}
\psi = \frac{p - p_0}{\rho_0 \| g\|}
\end{equation*}
is the {\em pressure head}. The compressibility models are usually further simplified to their first order Taylor series expansions
\begin{eqnarray*}
\rho(\psi) &=& \rho_0 \( 1 + c_0 \psi \) \\
\omega(\psi) &=& \omega_0 \( 1 + c_R \psi \) 
\end{eqnarray*}
The resulting model is
\begin{equation}
\framedEquation{
\pd{[\omega(u) \rho(u)]}{t} + \deld \left[ - \ten{a} ( u) \grad u + \vec b(u) \right] = 0 }
\end{equation}
where
\begin{equation*}
\vec b(u) = \frac{\ten{a} ( u)  \rho(u) \vec g}{\rho_0 \| g\|} 
\end{equation*}
and $u=\psi$. Often the spatial gradients of density and the compressibility in $\omega$ are neglected so we recover the heat equation
\begin{equation}
\framedEquation{
c \pd{u}{t} + \deld \left[ - \ten{a} \grad u \right] = 0 }
\end{equation}
where $u=\phi$.

\section{Immiscible Twophase Flow}
We will describe the flow of two phases (e.g. air and water) using the
water {\em saturation}, which is the fraction of the ``pore space''
containing water and a pressure that depends on the particular model
formulation. We denote these variables $u_s$ and $u_p$.  We say a
point in the medium is fully saturated when $u_s=1$ and dry when
$u_s=0$.

\subsection{Incompressible Case}

\begin{equation}
\framedArray{
\pd{u_s}{t} + \deld \left[ - \ten{a}_w ( u_s) \grad u_p + \vec b_w(u_s) \right] &=& 0 \\
-\pd{u_s}{t} + \deld \left[ - \ten{a}_n ( u_s) (\grad u_p + \ten{c}(u_s) \grad u_s) + \vec b_n(u_s) \right] &=& 0 }
\end{equation}
where $\ten{a}_w$ and $\ten{a}_n$ are the wetting and non-wetting
hydraulic conductivities and $\ten{c}$ is an additional nonlinear closure
relation. All of the nonlinearities in multiphase flow are continuous,
but they can be non-Lipschitz at a single point corresponding to
$u_s=1$ in some important cases. Furthermore $\ten{a}_n = 0$ at
$u_s=1$ and $\ten{a}_w \rightarrow 0 $ as $u_s \rightarrow 0$.

\subsubsection{Fractional Flow Form}

Adding the above equations and making some reasonable assumptions
about the form of the $\vec x$ and $u$ dependence of the coefficients
we obtain a different formulation called the {\em fractional flow}
formulation given by
\begin{equation}
\framedArray{
\pd{u_s}{t} + \deld \left[ - \ten{a}(u_s) \grad u_s + b(u_s) \vec q_t \right] &=& 0  \label{buckley}\\
\vec q_t + \ten{a}_t(u_s) ( \grad u_p - \vec b_t(u_s) )  &=& 0\\
\deld \vec q_t &=& 0 }
\end{equation}
Again the nonlinearities are continuous but possibly non-Lipschitz.
The pressure variable $u_p$ now corresponds to a special kind of
average of the individual phase pressures that comes from the
derivation of the fractional flow formulation.  Furthermore $\vec b(u_s)$
(the fractional flow) is non-convex and non-monotone and $\ten{a}(u_s)
= 0$ at $u_s=1$. In one-dimension $\vec q_t$ is constant in space so
we need only solve the first equation of \ref{buckley}, which is known as the
Buckley-Leverett equation.

\subsection{Compressible Case} 
Using the same density and porosity models as with the saturated case we get
\begin{equation}
\framedArray{
\pd{[m_w(u_s,u_p)]}{t} + \deld \left[ - \ten{a}_w ( u_s,u_p) \grad u_p + \vec b_w(u_s,u_p) \right] &=& 0 \\
\pd{[m_n(u_s,u_p)]}{t} + \deld \left[ - \ten{a}_n ( u_s,u_p) (\grad u_p + \ten c(u_s) \grad u_s) + \vec b_n(u_s,u_p) \right] &=& 0 }
\end{equation}
where we have simply used nonlinear bulk densities $m_w,m_n$ to simplify notation.
\subsubsection{Fractional Flow Form}
The fractional flow formulation becomes
\begin{equation}
\framedArray{
\pd{[m_w(u_w,u_p)}{t} + \deld [  - \ten{a}(u_w,u_p) \grad u_s + \vec b(u_s,u_p) \vec q_t  ] &=& 0 \\
\pd{m_t(u_w,u_p)}{t} - \deld  \vec q_t &=& 0 \\
\vec q_t + \ten{a}_t(u_w,u_p)  (\grad u_p + \vec b_t(u_s,u_p) ] &=& 0\\ }
\end{equation}

\subsection{Richards Equation}

A related model for flow of water in a variably saturated porous
medium is called Richards equation
\begin{equation}
\framedEquation{
\pd{[m_w(u_p)]}{t} + \deld \left[ - \ten{a}_w ( u_p) \grad u_p + \vec b_w(u_p) \right] = 0 }
\end{equation}
Richards derived his equation based on the physics of flow in
capillary tubes, but it can be seen as a simplification of the full
two-phase model. This model has continuous coefficients but $\ten{a}_w$ can be non-Lipschitz at $u_p=0$. Also $\ten{a}_w \rightarrow 0$ as $u_p \rightarrow - \infty$

\subsection{Kirchof Transformation}

We can rewrite terms like $a(u) \grad u$ as $\grad \phi$ where 
\begin{equation*}
\phi = \int_0^u a(v) dv
\end{equation*}
this trick is called the Kirchof transformation and $\phi(u)$
the Kirchof potential. If $\phi(u)$ is invertible then we can use
$\phi$ as the solution variable and there will be no nonlinearity in
the second order term. For instance, Richards' equation becomes
\begin{equation}
\framedEquation{
\pd{[m_w(\phi)]}{t} + \grad^2 \phi +  \deld \vec b_w(\phi)  = 0 
}
\end{equation}


\section{Transport}
We will assume that the chemical being transported in the
fluid does not effect the bulk fluid phase flow.  Transport of a
chemical in one of the phases is then
\begin{equation}
\framedEquation{
\pd{m(u)}{t} + \deld [- \ten{a} \grad u + \vec{f} u ]  +  r(u) = 0
}
\end{equation}
where $\ten{a}$ is the symmetric positive definite diffusion coefficient, $\vec{f}$ is the fluid
velocity, $r$ is a reaction term and $m$ may represent reactions that
occur fast enough with respect to the other terms that they can be
considered to have reached equilibrium instantaneously. The reactions $m$ and $r$ can be non-Lipschitz. 
\end{document}


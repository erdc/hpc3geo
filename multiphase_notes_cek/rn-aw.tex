\documentclass[10pt,dvips,twoside,reqno]{amsart} 
\usepackage{epsfig}
\usepackage[authoryear]{natbib} \setlength{\oddsidemargin}{0in}
\setlength{\evensidemargin}{0in} \setlength{\topmargin}{-0.25in}
\setlength{\headheight}{0.1in} \setlength{\headsep}{0.15in}
\setlength{\topskip}{0in} \setlength{\footskip}{0.15in}
\setlength{\textwidth}{6.5in} \setlength{\textheight}{9in}

%differential operators
\newcommand{\grad}{\nabla}
\newcommand{\deld}{\nabla \cdot}
\newcommand{\lap}{\Delta}
%boldface in math mode
\newcommand{\bm}[1]{\mbox{{\boldmath ${#1}$}}}
% vectors and tensors
\renewcommand{\vec}[1]{{\bf #1}}
\newcommand{\gvec}[1]{\mbox{{\boldmath ${#1}$}}}
\newcommand{\ten}[1]{\bar{\bm{#1}}}
%derivatives
\newcommand{\od}[2]{\frac{d {#1}}{d {#2}}}
\newcommand{\ods}[2]{\frac{d^2{#1}}{d {{#2}^2}}}
\newcommand{\pd}[2]{\frac{\partial {#1}}{\partial {#2}}}
\newcommand{\pds}[2]{\frac{\partial^2{#1}}{\partial {{#2}^2}}}
\newcommand{\pdsm}[3]{\frac{\partial^2{#1}}{\partial {#2}\,\partial {#3}}}
%funtional analysis
\newcommand{\abs}[1]{\left| #1 \right|}
\newcommand{\norm}[1]{\left\| #1 \right\|}
\newcommand{\iprod}[2]{\left( #1, #2 \right)}
\newcommand{\dprod}[2]{\left\langle #1, #2 \right\rangle}
%real numbers
\newcommand{\field}[1]{\mathbb{#1}}
\newcommand{\R}{\field{R}}
%funciton spaces
\newcommand{\M}{\mathcal{M}}
%delimiters
\newcommand{\pl}{\left(}
\newcommand{\pr}{\right)}
\newcommand{\sbl}{\left[}
\newcommand{\sbr}{\right]}
\newcommand{\dbl}{\left[\hspace{-0.05cm}\left[}
\newcommand{\dbr}{\right]\hspace{-0.05cm}\right]}
\newcommand{\cbl}{\left\{ }
\newcommand{\cbr}{\right\} }
\newcommand{\eqn}[1]{equation \ref {eq:#1}} 
\newcommand{\Eqn}[1]{Equation \ref {eq:#1}} 
\newcommand{\eqnst}[2]{equations \ref{eq:#1} and \ref{eq:#2}} 
\newcommand{\Eqnst}[2]{Equations \ref{eq:#1} and \ref{eq:#2}} 
\newcommand{\eqns}[2]{equations \ref{eq:#1}--\ref{eq:#2}} 
\newcommand{\Eqns}[2]{Equations \ref{eq:#1}--\ref{eq:#2}}
\newcommand{\msection}[1]{ \vspace{.2in} {\noindent \bf #1}.}
\renewcommand{\for}{\mbox{for}\quad}
%\newcommand{\for}{\mbox{for}\quad}
\newcommand{\argmin}{\mbox{argmin}}
\newcommand{\argmax}{\mbox{argmax}}
\newcommand{\fig}[1]{figure \ref{fig:#1}} 
\newcommand{\Fig}[1]{Figure \ref{fig:#1}} 
\newcommand{\figst}[2]{figures \ref {fig:#1} and \ref {fig:#2}} 
\newcommand{\Figst}[2]{Figures \ref {fig:#1} and \ref {fig:#2}} 
\newcommand{\figs}[2]{figures \ref{fig:#1}--\ref{fig:#2}} 
\newcommand{\Figs}[2]{Figures \ref{fig:#1}--\ref{fig:#2}}
\newcommand{\tab}[1]{table \ref {tab:#1}} 
\newcommand{\Tab}[1]{Table \ref {tab:#1}} 
\newcommand{\tabst}[2]{tables \ref {tab:#1} and \ref {tab:#2}} 
\newcommand{\Tabst}[2]{Tables \ref {tab:#1} and \ref {tab:#2}} 
\newcommand{\tabs}[2]{tables \ref{tab:#1}--\ref{tab:#2}} 
\newcommand{\Tabs}[2]{Tables \ref{tab:#1}--\ref{tab:#2}}
\newtheorem{theorem}{Theorem}
\newenvironment{neqnarray}[1]{\begin{minipage}[t]{6.5in}  \begin{minipage}[b]{1.0in} #1 \end{minipage}  \begin{minipage}[b]{5.5in}\begin{eqnarray}}{\end{eqnarray}\end{minipage}\end{minipage}}
\newcommand{\bneqnarray}[2]{\\ \\ \fbox{\begin{neqnarray}{#1} #2 \end{neqnarray}}\\ \\ \noindent} 

\begin{document}


\begin{center}
  {\bf NOTES ON AIR/WATER FLOW IN POROUS MEDIA} \\
  Chris Kees\\
  Coastal and Hydraulics Laboratory,
  U.S. Army Engineer Research and Development Center,\\
  Vicksburg, MS 39180, email: \texttt{christopher.e.kees@erdc.usace.army.mil} \\
  October 26, 2005
\end{center}
\markboth{NOTES ON AIR/WATER FLOW IN POROUS MEDIA/KEES}{NOTES ON AIR/WATER FLOW IN POROUS MEDIA/KEES}

\begin{center} TODO \end{center}
\begin{enumerate}
\item CMWR Abstract: Richards' equation and the two-phase flow
  equations are well-known degenerate parabolic models of air/water
  flow in porous media. Poor iterative solver performance and small
  time steps during transient simulations are often reported in
  field-scale simulations. In this work we study Newton-multigrid and
  nonlinear multigrid methods applied to discrete air/water flow
  models. The models are discretized using standard continuous finite
  element spaces. Due to strong nonlinearity and potential degeneracy
  in the coefficients, we stabilize the models using a multiscale
  approach. We present computational results comparing iterative
  solver perfomance and solution accuracy, focusing particularly on
  the effects of degenerate coefficients in wetting and drying
  problems.
\item Matthew's and Lea's abstract
\item Multiscale/shock capturing FEM (done, need to write up).
\begin{enumerate}
\item multiscale stabilization
\item shock capturing numerical diffusion
\item BDF1 or BDF-k 
\item mass conservation flux calculation
\end{enumerate}
\item DG/MHFEM
\item Show eigenvalues versus $h$ (done in 1D for simple test problems).
\item Write up multigrid algorithms.
\begin{enumerate}
\item MGM
\item NI(MGM)=FMG
\item Newton-NI(MGM), NI-Newton-NI(MGM)
\item Add GMRES layer or just terminate on residuals? How to choose residual tolerance(fraction of nonlinear tolerance, which should be h-dependent, I think)?
\end{enumerate}
\item Figure out how to present multigrid results, also time-dependent case.
\item Maybe go back and put in erfc solutions and Laplace transform/semigroup approach to look at eigenvalues.
\item See if Barenblatt solution can be fixed for nonlinear advection as well.
\item Think about whether finite time to equilibrium is really the right in fast diffusion case.
\item Fix EVec and Mesh implementation problems w.r.t. memory and speed.
\item Peel off GraphTools for view/write
\end{enumerate}


\tableofcontents 

\section{Model Formulations}

\subsection{Primitive Continuuum Model Equations}

For a mixture of two mobile, immiscible fluid phases (labeled wetting,
$w$, and a non-wetting, $n$) the standard model is
\begin{eqnarray}
\pd{(\omega s_w \rho_w)}{t} + \deld (\rho_w \vec v_w) +c_w&=& d_w \label{eq:satW} \\
\pd{(\omega s_n \rho_n)}{t} + \deld (\rho_n \vec v_n) +c_n&=& d_n \label{eq:satN}\\
\vec v_w &=& - \frac{\ten{k}_i \ten{k}_{rw}}{\mu_w}(\grad p_w - \rho_w \vec g)  \label{eq:waterFlux} \\
\vec v_n &=& - \frac{\ten{k}_i \ten{k}_{rn}}{\mu_n}(\grad p_n - \rho_n \vec g)  \label{eq:nwFlux}\\
s_n &=& 1 - s_w \label{eq:sat_const} \\
p_n - p_w &=& p_c(\vec x,s_w) \label{eq:pc_S} \\
\ten k_{r \alpha} &=& \ten k_{r \alpha}(\vec x,s_w) \label{eq:kr_S} \\
\ten k_{i} &=& \ten k_{i}(\vec x) \\
\rho_{\alpha} &=& \rho_{\alpha}(p_{\alpha}) \\
\omega &=& \omega(\vec x,p_n,p_w) \\
\mu_{\alpha} &=& \mu_{\alpha}(p_{\alpha})
\label{eq:viscosity} 
\end{eqnarray}

\subsection{Working Formulations}

I'll adopt the following dimensionless variables.
\bneqnarray{Dimensionless Variables}{
  \tilde{t} &=& t \sqrt{\frac{g}{l}} \\
  \tilde{\vec x} &=& \vec x \frac{1}{l} \\
  \tilde{\psi}_{\alpha} &=& \frac{p_{\alpha} - p_{w,0}}{\rho_{w,0} g l} \label{eq:head2p}\\
  \tilde{\psi}_c &=& \tilde{\psi}_n - \tilde{\psi}_w \label{eq:capHead}\\
  \tilde{\rho}_{\alpha} &=& \frac{\rho_{\alpha}}{\rho_{\alpha,0}} \\
  \tilde{\ten K} &=& \frac{\rho_{w,0} g \ten k}{\mu_w \sqrt{gl}} \\
  \tilde{\vec v}_{\alpha} &=& \vec v_{\alpha} \frac{1}{\sqrt{gl}} \\
  \tilde{\vec g} &=& \frac{\vec g}{g} \label{eq:dg2p} \\
  \tilde{z} &=& -\tilde{\vec g} \cdot (\tilde{\vec x} - \tilde{\vec x}_0) \\
  \tilde{\mu_{\alpha}} &=& \frac{\mu_{\alpha}}{\mu_w}\\
  \tilde{\lambda}_w &=& \frac{\tilde{\rho}_w k_{rw}}{\tilde{\mu}_w}\\
  \tilde{\lambda}_n &=& \frac{\tilde{\rho}_n k_{rn}}{\tilde{\mu}_n}\\
  \tilde{c}_{\alpha} &=& \frac{c_{\alpha}}{\rho_{\alpha,0}}\sqrt{\frac{l}{g}} \\
  \tilde{d}_{\alpha} &=& \frac{d_{\alpha}}{\rho_{\alpha,0}}\sqrt{\frac{l}{g}} \\
  b &=& \frac{\rho_{n,0}}{\rho_{w,0}}} 
Henceforth we drop the tilde. Skipping a lot of details, we come
to the following set of model formulations that are useful for
analysis and computations. We write the general two-phase system as a
general advection-diffusion-reaction system with unknowns $u_1,u_2$
and nonlinear coefficients $m,f,a,\phi,r$
\begin{eqnarray}
m_{1,t} + \deld \pl \vec f_1 - \sum_{j=1}^2 \ten{a}_{1,j} \grad \phi_{1,j} \pr + r_1 = 0 \\
m_{2,t} + \deld \pl \vec f_2 - \sum_{j=1}^2 \ten{a}_{2,j} \grad \phi_{2,j} \pr + r_2 = 0 
\end{eqnarray}
The terms $r_1$ and $r_2$ are problem dependent sources. We introduce the following auxiliary variables 
\begin{eqnarray}
\lambda_t &=&  \lambda_w + \lambda_n \\
f_n &=& \frac{\lambda_n}{\lambda_t}\\
f_w &=& \frac{\lambda_w}{\lambda_t}\\
\rho_t &=& f_w\rho_w  + f_n b \rho_n = \rho_w + f_n (b \rho_n - \rho_w)\\
\vec q_t &=& \rho_n \vec v_n + \rho_w \vec v_w
\end{eqnarray}
The first working formulation is the primitive form, which defines the coefficints of the system as
\bneqnarray{2p-$s_w$-$\psi_w$}{
m_{1} &=& \omega \rho_w s_w \\
m_{2} &=& \omega \rho_n (1-s_w) \\
f_1 &=& - \rho^{2}_{w} \ten{K} \lambda_w \grad z \\
f_2 &=& - \rho^{2}_{n} \ten{K} \lambda_n b \grad z \\
\ten{a}_{1,1} &=& \rho_w \ten{K} \lambda_w\\
\ten{a}_{2,2} &=& \rho_n \ten{K} \lambda_n\\
\phi_{1,1} &=& \psi_w \\
\phi_{2,2} &=& \psi_c(s_w) + \psi_w\\
u_1 &=& s_w \\
u_2 &=& \psi_w }
For the next two formulations I'll neglect compressibility and some
forms of heterogeneity. If compressibility and heterogeneity are
present some additional ``correction'' terms are required in the
equations.  The second working form is the fractional flow with phase
pressure
\bneqnarray{2p-$s_w$-$\psi_w$-FF}{
m_{1} &=& \omega \rho_w s_w\\
m_{2} &=& \omega \sbl s_w (\rho_w - \rho_n) + \rho_n \sbr\\
f_{1} &=& f_w q_t + \ten{K} \lambda_w f_n (b \rho_n - \rho_w) \grad z \\
f_{2} &=& \ten{K} \lambda_t \rho_t \grad z\\
a_{1,1} &=& \ten{K} \\
a_{2,1} &=& \ten{K} \\
a_{2,2} &=& \ten{K} \lambda_t \\
\phi_{1,1} &=& \int^1_{s_w} \lambda_w f_n \od{p_c}{s_w} ds\\
\phi_{2,1} &=& \int^1_{s_w} \lambda_n \od{p_c}{s_w} ds\\
\phi_{2,2} &=& \psi_w \\ 
u_1 &=& s_w \\
u_2 &=& \psi_w}
The third working form is the fractional flow with total (global)
pressure \bneqnarray{2p-$s_w$-$\psi_t$-FF}{
  m_{1} &=& \omega \rho_w s_w\\
  m_{2} &=& \omega \sbl s_w (\rho_w - \rho_n) + \rho_n \sbr\\
  f_{1} &=& f_w q_t + \ten{K} \lambda_w f_n (b \rho_n - \rho_w) \grad z \\
  f_{2} &=& \ten{K} \lambda_t \rho_t \grad z\\
  a_{1,1} &=&  \ten{K} \\
  a_{2,2} &=&  \ten{K} \lambda_t \\
  \phi_1 &=& \int^1_{s_w} \lambda_w f_n \od{p_c}{s_w} ds_w\\
  \phi_2 &=& \psi_t = \psi_w + \int_{-\infty}^{\psi_c}  f_n(w) dw = \frac{\psi_w + \psi_n}{2} + \int^{1}_{s_e} \pl f_w - \frac{1}{2} \pr \od{\psi_c}{s_e} dw\\
  u_1 &=& s_w \\
  u_2 &=& \psi_t }
I give two formulations of the scalar Richards' equation.
First the pressure head equation in conservative form
\bneqnarray{Richards-$\psi$}{
  m_1 &=& \omega \rho_w s_w\\
  f_1 &=& -\ten{K} \lambda_w \rho_w \grad z \\
  a_1 &=& \ten{K} \lambda_w \\
  \phi_1 &=& \psi_w \\
  u_1 &=& \psi_w} Lastly, a transformed Richards' equation in conservative form that is an extension of the saturation-based Richards' equation.
\bneqnarray{Richards-u}{
  m_1 &=& \omega \rho_w s_w\\
  f_1 &=& -\ten{K} \lambda_w \rho_w \grad z \\
  a_1 &=& \ten{K} \lambda_w \\
  \phi_1 &=& \cbl \begin{array}{lc} \int_0^{u_1} \lambda_w \od{p_c}{s_w} \od{s_w}{u_1}du_1 & u_1 < 0 \\
u_1  &  u_1 \geq 0 \end{array} \right.\\
s_w &=& 1-C \max(-u_1,0)^p \\
\psi_w &=& \cbl \begin{array}{lc}  - \psi_c(s_w(u_1)) & u_1 < 0 \\
u_1 & u_1 \geq 0  \end{array} \right. }
The constants $C$ and $p$ are chosen (if possible) so that
\begin{equation}
\lim_{u_1 \rightarrow 0^{-}}  \lambda_w \od{p_c}{s_w} \od{s_w}{u_1} = 1 
\end{equation}
hence $\phi_1(u_1)$ is $C^1$, in particular, $\od{\phi_1}{u_1}(0) =
1$. The transformation depends on the constitutive relations and it
may be necessry to relax $C^1$ to Lipschitz
continuous (see later section).

The constitutive equations are summarized next
\begin{center}
\fbox{\begin{minipage}[b]{6.5in}
\begin{align}
  \intertext{p-s-k Closure Relations}
  s_e &= \frac{s_w -  s_m(\vec x)}{s_M(\vec x) - s_m(\vec x)} && \mbox{(effective saturation)}\\
  p_{ce} &= \frac{p_c}{p_{cM}(\vec x)} && \mbox{(effective capillary pressure)}\\
  \intertext{\center Simple Model}
  p_{ce} &= s_e \\
  k_{rw} &= s_e^2  \\
  k_{rn} &= \frac{1}{
2}(1-s_e)^2
  \intertext{\center Van Genuchten-Mualem}
  p_{cM} &= \frac{1}{\alpha(\vec x)} && \alpha > 0\\
  p_{ce} &= \pl s_e^{-1/m} - 1 \pr^{1-m} && 0 < m < 1\\
  k_{rw} &= s_e^{1/2} \sbl 1-(1-s_e^{1/m})^m \sbr^2 \\
  k_{rn} &= (1-s_e)^{1/2}(1-s_e^{1/m})^{2m} \\
  \intertext{\center Brooks-Corey-Burdine}
  p_{cM} &= p_d(\vec x) &&p_d > 0\\
  p_{ce} &= \Big\{ \begin{array}{ll}
    s_e^{-1/\lambda} & s_e < 1   \\
    0 & s_e = 1 
  \end{array} && \lambda > 0\\
  k_{rw} &= s_e^{(2 + 3 \lambda)/\lambda} \\
  k_{rn} &= (1-s_e)^2(1-s_e^{(2+\lambda)/\lambda}) 
\intertext{Density Closure Relations (Dimensionless)}
\rho &= 1 &&\mbox{(constant)}\\
\rho &= e^{\beta_f \psi} &&\mbox{(exponential)}\\
\rho &= 1 + \beta_f \psi &&\mbox{(linear)}\\
\rho &= p/Z(p,T)RT &&\mbox{(real gas,Z=1 for ideal)}\\
\rho &= (\psi/\psi_0)^m e^{\beta_f \psi} &&\mbox{(general)}\\
\intertext{Porosity Closure Relations (Dimensionless)}
\rho &= \omega_0 &&\mbox{(constant)}\\
\rho &= 1 + \beta_m \psi &&\mbox{(linear)}
\end{align}
\end{minipage}}
\end{center}

\begin{center}
\fbox{\begin{minipage}[b]{6.5in}
\begin{align}
\intertext{Dimesional Quanties (Examples--Source http://www.hbcpnetbase.com)}
\rho_0 &= 997.0 && [kg/m^3] \mbox{ for water at $25^\circ$ C and 101325Pa}\\
\rho_0 &= 1.205 && [kg/m^3] \mbox{ for air at $20^\circ$ C and 101325Pa}\\
\beta_f &= \rho_0 g l \cdot 4.524\times10^{-10} &&[(kg \cdot m /s^2)^{-1} = Pa^{-1}]  \mbox{ for water at $25^\circ$ C}\\
\beta_f &= \rho_0 g l \cdot9.87\times10^{-6} &&[(kg \cdot m /s^2)^{-1} = Pa^{-1}]  \mbox{ for air at $25^\circ$ C}\\
\beta_m &= \rho_0 g l \cdot 10^{-6} \mbox{---} 10^{-11} &&[(kg \cdot m /s^2)^{-1} = Pa^{-1}]  \mbox{ for various media}\\
\mu &= 8.9\times 10^{-4} && [kg / m \cdot s = Pa \cdot s] \mbox{ for water at $25^\circ$} \\
\mu &= 7.1\times 10^{-6} && [kg / m \cdot s = Pa \cdot s] \mbox{ for air at $25^\circ$} \\
g &= 9.8 && [m/s^2] \mbox{ for Mississippi} \\
\ten{k}&= 10^{-7}\mbox{---}10^{-16} && [m^2] \mbox{ for various media}\\
\omega&=0.01\mbox{---}0.9 && [-] \mbox{ for various media} \\ 
R&=8.314 &&[kg \cdot m^2 / K mol s^2]
\end{align}
\end{minipage}}
\end{center}

\subsection{Asymptotic Behavior of Coefficients and Simplified Test Problem}

In this section my objectives are to motivate a simple test problem
and to derive the variable transformations that are required for some
cases, in particular, the last form of Richards' equation that I
presented above. Depending on the constitutive relations, we have
saturation dependent coefficients that may have derivatives that blow
at $s_e=1$. To look at the asymptotic behavior we change variables to
$u=1-s_e$ (i.e., effective non-wetting phase saturation). We have as $u
\rightarrow 0$ for the Van Genuchten-Mualem relations
\begin{eqnarray}
\od{p_c}{s} &=& \frac{m-1}{m}\frac{1}{m}^{-m} u^{-m} + o(u^{-m}) \label{eq:pcApp}\\
k_{rw} &=& 1-2 \frac{1}{m}^m u^m + o(u^m) \label{eq:kwApp} \\
k_{rn} &=& \frac{1}{m}^{2m} u^{\frac{1}{2}+2m} + o(u^{\frac{1}{2}+2m}) \label{eq:knApp}
\end{eqnarray}
Neglecting constants, we can approximate two-phase flow with the nonlinearities
\begin{eqnarray}
  \label{eq:twopSimpCoef}
  \od{\phi_{1,1}}{s_w} &=& -\frac{k_i}{\omega \mu_w}\frac{m-1}{m}\frac{1}{m}^{m} u^{\frac{1}{2} + m}  + o(u^{\frac{1}{2} + m}) \approx   u^{\frac{1}{2}+m} \label{eq:blCoeffDapp}\\
  f_1 &=& \frac{k_i}{\omega \mu_w} \frac{1}{m}^{2m} u^{\frac{1}{2}+2m} (1 - \frac{\rho_n}{\rho_w}) \rho_wg_x + o(u^{\frac{1}{2}+2m}) \nonumber \\
&\approx&   u^{\frac{1}{2}+2m} \label{eq:blCoeffFapp}
\end{eqnarray}
and Richards' equation with
\begin{eqnarray}
  \od{\phi_1}{s_w} &=& \frac{k_i}{\omega \mu_w}\frac{m-1}{m}\frac{1}{m}^{-m} u^{-m}+ o(u^{-m}) \approx   u^{-m} \label{eq:reCoeffDapp}\\
  f &=& -\frac{k_i}{\omega \mu_w} \pl 1-2 \frac{1}{m}^m u^m\pr \rho_w  g_x  + o(u^m) \nonumber \\
&\approx&   1-u^m \label{eq:reCoeffFapp}
\end{eqnarray}
Making these approximations yields a simple model
\begin{equation}
  u_t + (b u^p - a u^q_x)_x = 0 \label{eq:simpMod}
\end{equation}
where $b =1$ for two-phase flow and $b=-1$ for Richards' equation.

In terms of the parameter $m$, the approximate two-phase flow model
has $q = 3/2 + m$ and $p=1/2+2m$ where the ranges of $p$ and $q$ are
$1<q<5/2$ and $0<p<5/2$. Thus, $f$ has infinite slope at $s=1$ for
$m<1/4$, which holds both for the simplified model and the original
two-phase model. The approximate Richards' equation has $q=1-m$ and $p
= m$ where the range of $p$ and $q$ is $0<p,q<1$. Thus $|\od{f}{u}|$
blows up faster than $\od{\phi}{u}$ when $m < 1/2$.

\begin{table}
\caption{Summary of model parameters for $0<m<1$ \label{tab:pqrParms}}
\begin{tabular}{lccc}
&$q$ & $p$ & $b_0$ \\
\hline
Two-phase & 3/2 + m & 1/2 + 2m & -1 \\
Richards & 1-m & m & 1 \\ 
\hline
\end{tabular}
\end{table}

\subsection{Variable Transformation}

The asymptotic approximations above not only yield simplified model
equations but also a change of variables for the original
model equations that will yield $C^1$
coefficients over the entire range of saturations $0\leq s_e \leq 1$. 

For two-phase flow in the case $m<1/4$, we will make the substitution
\begin{equation}
  \label{eq:cov}
  v = \frac{1}{m}^{2m} u^{\frac{1}{2}+2m}
\end{equation}
This transformation yields nonlinear coefficients $s$, $f$, and $\phi$
that are differentiable in $v$.

The variable transformations are slightly more complex for Richards'
equation. For $m\geq1/2$, $\od{\phi}{u} \rightarrow \infty$ and $\od{\phi}{u}/|\od{f}{u}|
\rightarrow \infty$ as $u \rightarrow 0$.  We design a change of
variables $v(u)$ so that $(dp_c/ds)(d v/du)(0) = 1$. Such a $v(u)$
will yield a diffusion coefficient that is bounded and continuous
and, therefore, a potential $\phi$ that is differentiable and an
advective flux that is zero at $u=0$.  This consideration leads to
\begin{equation}
  \label{eq:cofRE1}
  v = -m^m u^{1-m}
\end{equation}
For $m<1/2$, $\od{\phi}{u}/|\od{f}{u}| \rightarrow 0$ as $u \rightarrow 0$. Thus,
as we did in the two-phase case, we construct a change of variables so
that the advective flux is differentiable:
\begin{equation}
  \label{eq:covRE2}
  v = -2 \frac{1}{m}^m u^m
\end{equation}

For the simplified model equation we can use 
\begin{equation}
  \label{eq:pqr}
  v = u^{\min(p,q)}
\end{equation}

If we don't use operator splitting for the pressure equation in
two-phase flow we may have trouble with the $\lambda_t$ term in
$a_{2,2}$, which is not differentiable with respect to $s_w$ for these
closure relations. We'll either have to use operator splitting,
neglect those terms in the Jacobian, or choose a variable
transformation that ``kills off'' those terms too.

\section{Analytical Solutions \label{analytical}}

In the following sections we look at analytical solutions that may be
useful in understanding the behavior of two-phase flow or our
numerical methods. I'm mainly interested in solutions with simple
algebraic representations.

\subsection{Differential-Algebraic Equations}

To test out time integration methods and get a feel for some of the
dynamics we study the initial-value problem
\begin{eqnarray}
\od{u}{t} &=& - a \max(u,0)^p \\
u(0) &=& 1
\end{eqnarray}
where $p > 0$. A solution can be obtained by seperation of variables
\begin{equation}
u(t) = \cbl \begin{array}{ll} \max(1 - (1-p) a t,0)^{\frac{1}{1-p}} & p \neq 1 = f(u) \\
e^{-a t} & p = 1 \end{array} \right.
\end{equation}
The equilibrium value of $u$ is zero. If $p<1$, the solution reaches
equilibrium in ``finite'' time. We will see similar behavior in the
``fast diffusion'' case of the the PDE models, which are characterized
by certain portions of the domain reaching equilibrium in finite time.
If $p > 1$ the solution tends to equilibrium slower than the standar
linear model $p=1$.  For all cases the soluions are $C^1$; however,
for $p<1$ the right hand side, $f(u)$ is not Lipschitz continous and
thus the standard proof for uniqueness of solutions to initial value
problems does not hold. Furthermore if we look at a perturbation $u +
\delta$ and linearize the right hand side about $u$ we have the linear
equation for the perturbation
\begin{equation}
\od{\delta}{t} = - a p \max(u,0)^{p-1} \delta = - \lambda \delta
\end{equation}
Thus we suspect that for $p<1$ only methods that are linearly stable
in the case $\lambda \rightarrow \infty$ would be stable for this
equation. On the other hand for $p > 1$, conditionally stable methods
might be sufficient.

For $p<1$, we think of the problem as a semi-explicit DAE
\begin{eqnarray}
\od{u}{t}&=& -\lambda v = f(v)\\
0 &=& u - \max(v,0)^{\frac{1}{p}} = g(u,v)
\end{eqnarray}
One differentian of the constraint with respect to time yields
\begin{eqnarray}
0 &=& \od{u}{t} - \frac{1}{p} \max(v,0)^{\frac{1-p}{p}} \od{v}{t} \\
&=& -\lambda v - \frac{1}{p} \max(v,0)^{\frac{1-p}{p}} \od{v}{t} \\
\end{eqnarray}
which yields, for $u \neq 0$, and after substituting $-\lambda v =
\od{u}{t}$ and $u = \max(v,0)^{\frac{1}{p}}$
\begin{equation}
\od{v}{t} = - p \lambda \frac{v^2}{u} 
\end{equation}
Hence, away from $u = 0$ the equation is uniformly index 1. At $u=0$
we have the ``hidden'' constraint
\begin{equation}
0 = -\lambda v = g_u(u,0) f(v) 
\end{equation}
Differentiating again yields
\begin{equation}
0= -\lambda v_t 
\end{equation}
\subsection{Linear Time-Dependent Advection-Diffusion-Reaction Equation}

We consider the Cauchy problem for the linear model problem
\begin{eqnarray}
\pd{u}{t} +(b u_x - a u_x)_x + c u &=& 0 \\
u(x,0) &=& g(x)
\end{eqnarray}
Where $g$ has compact support. Changing coordinates to $\xi = x - bt$
and employing seperation of variables yields a family of solutions
$u_n = v_n(t) w_n(\xi)$ where
\begin{eqnarray}
v_n(t) &=& C_n e^{-\sbl a\pl n \pi \pr^2 + c \sbr t} \\
w_n(\xi) &=& \sin(n \pi \xi)
\end{eqnarray}
The theory of Fourier series gives existence of $C_n$ such that
\begin{equation}
g(x) = \sum_{n=1}^{\infty} C_n \sin(n \pi x)
\end{equation}
with convergence of the series being point wise. Linearity of the
equation (principle of superposition) then gives the solution
\begin{equation}
u(x,t) = \sum_{n=1}^{\infty} u_n(x,t)
\end{equation}
where term by term differentiation is valid. Each ``mode'' has the
form
\begin{equation}
u_n(x,t) = C_n e^{-\sbl a\pl n \pi \pr^2 + c \sbr t} \sin\sbl n \pi \pl x - bt\pr \sbr
\end{equation}
This solution shows that all modes are damped as long as $a\pl n \pi
\pr^2 + c > 0$, and the higher the frequence, the stronger the
damping.  Also, for $b=0$ the $a \pl n \pi \pr^2 + c$ are the
eigenvalues of the linear differential operator $L$ where the
continuous evolution equation is $u_t = -L u$. If $L_h \rightarrow L$
for some discrete operator $L_h$ then we see that $L_h$ must have a
condition number $\lambda_{h,max}/\lambda_ {h,min}$ that rapidly
worsens (if convergence to $L$ is rapid). On the other hand if $b \neq
0$ the damping is unchanged, but the solution also oscillates in time.
A different analysis yields that the eigenvalues of $L$ in that case
are complex, but still have a positive real part.

\subsection{Steady-State Equation}

Consider the nonlinear boundary value problem
\begin{equation}
(b u^p - a u^q_x)_x = 0 \quad u(0) = 1; u(1) = 0
\end{equation}
Changing variables to $w=u^q$, integrating once, and collecting terms in the constant yields
\begin{equation}
w_x = \frac{b}{a}(w^{p/q} + C)
\end{equation}
Taking $w(0)=1$ as the initial condition for this equation and noting
that the right hand side is monotone for positive $w$, we see that for
the other boundary condition to be met we need, if $b>0$, $C<-1$,
which brackets $C$ on a half line (if $b<0$ we need $C > -1$. We'll
take $b>0$.  Integrating once again over the interval (x,1) yields
\begin{equation}
\int \frac{1}{w^{p/q} + C} dw = \frac{b}{a}(1 - x)
\end{equation}  
We consider several special cases. For $p=q=1$ we obtain
\begin{equation}
u = \frac{e^{\frac{b}{a}} - e^{\frac{b}{a} x}}{e^{\frac{b}{a}} - 1}
\end{equation}
For $p=2,q=1$ (fast diffusion) we obtain
\begin{eqnarray}
u &=& \sqrt{-C} \mbox{tanh}\sbl \frac{b\sqrt{-C}}{a} \pl 1 - x \pr \sbr \\
0 &=&  \sqrt{-C} \mbox{tanh}\sbl \frac{b\sqrt{-C}}{a}  \sbr - 1
\end{eqnarray}
For $p=1,q=2$ (fast diffusion) we obtain
\begin{eqnarray}
x &=& 1 - \frac{2a}{b}\sbl C \log \pl \frac{-C}{-u - C} \pr + u\sbr \\
0 &=& 1 - \frac{2a}{b}\sbl C \log \pl \frac{-C}{-1 -C} \pr + 1\sbr
\end{eqnarray}

\subsection{ First-Order Approximation}

First we consider solutions of the first-order approximation of
\begin{equation}
  \label{eq:bl}
  m_t + f_x = 0
\end{equation}
on the spatial domain $(-\infty,\infty)$ given Riemann initial data,
\begin{eqnarray}
\label{eq:riemannData}
u(x,0) = \left\{ \begin{array}{lr}
u_- & \for x < 0 \\
u_+ & \for x \geq 0
\end{array} \right.
\end{eqnarray}
\Eqn{bl} is usually referred to as the Buckley-Leverett equation when
$u=s$ and $f$ and $m$ are taken from the saturation equation of the
two-phase fractional flow formulation. This equation generally only
has weak solutions, which may be discontinuous and which furthermore
are not unique (c.f. \citep{Evans_98,Leveque_90}). When the solution
fails to be continuous, a discontinuity with left state $u_-$ and
right state $u_+$ must propagate with speed
\begin{equation}
  \label{eq:rhcondition}
  c = \frac{f(u_+) - f(u_-)}{m(u_+) - m(u_-)}
\end{equation}
Unique weak solutions can be specified by requiring that discontinuous
solutions be in some sense a limit of solutions of \eqn{bl} with a
second-order regularization term $\epsilon u_{xx}$ (a vanishing
viscosity solution).  For non-convex $f$, this requirement can be
stated as the Oleinik entropy condition \citep{Oleinik_57} (c.f.
\citep{Osher_84,Leveque_90})
\begin{equation}
  \label{eq:oleinik}
  \frac{f(u) - f(u_-)}{m(u) - m(u_-)} \geq c \geq  \frac{f(u) - f(u_+)}{m(u) - m(u_+)}
\end{equation}
for all $u$ between $u_-$ and $u_+$. For a general Riemann problem
with $m(u_-) < m(u_+)$, the unique solution can be determined from the
convex hull of the set
\begin{equation}
  \label{eq:chull}
  \cbl (m,y) | m(u_-) \leq m(u) \leq m(m_+) \mbox{ and } y \geq f(u) \cbr
\end{equation}
The convex hull will consist of the graph of $f(m)$ and some set of
chords enclosing regions where $f(m)$ is concave down. A chord on the
boundary of the convex hull represents two values of the solution
connected by a shock wave, and the slope of the chord is the shock
speed $c$. This solution can be computed automatically by solving the
minumization problem \citep{Osher_84}
\begin{equation}
  \label{eq:osherSol}
  u(\xi) = \argmin_{u \in [u_-,u_+]} \sbl f(u) - \xi m(u) \sbr 
\end{equation}
where $\xi = x/t$. This formula is a concise statement of the entropy
satisfying weak solution to the Riemann problem when $m_+ \geq m_-$ (a
similar formula holds for $m_+ \leq m_-$). This problem can be solved
in MATLAB using the \texttt{fminbnd} routine.

\subsection{Traveling Wave Solutions}

The solutions in the previous section are the unique entropy
satisfying weak solutions composed of continuous rarefaction waves and
discontinuous shock waves. The former condition arises due to the
conservation form of the equation while the latter can be derived, as
mentioned above, by requiring that any moving discontinuity correspond
to traveling wave solutions of a second-order regularization of the
equation. Now we look at traveling wave, which are the viscous analogs
of shock waves.

To derive traveling wave solutions, we assume that the solution has
the form $u(x,t) = u(x - ct) = u(\xi)$ where $c$ is the wave speed (to
be determined).
\begin{equation}
  \label{eq:travOdePrim1}
  -c m_\xi + f_\xi - \phi_{\xi \xi} = 0
\end{equation}
Integrating once with respect to $\xi$ and rearranging yields
\begin{equation}
  \label{eq:travOdePrim2}
  \phi_\xi  = -c m(\xi) + f(\xi) + B
\end{equation}
where $B$ is the constant of integration. We will enforce the
following asymptotic boundary conditions on \eqn{travOdePrim2}
\begin{eqnarray}
  \lim_{\xi \rightarrow \pm \infty} u &=& u_{\pm} \label{eq:travOdeData1}\\
  \lim_{\xi \rightarrow \pm \infty} \phi_\xi &=& 0  \label{eq:travOdeData2}
\end{eqnarray}
and write $f_{\pm}$ for $f(u_{\pm})$ and $m_{\pm}$ for $m(u_{\pm})$.
These boundary conditions require
\begin{equation}
  \label{eq:travLim}
  \lim_{u \rightarrow u_{\pm}} -c m + f + B = 0
\end{equation}
which implies
\begin{equation}
  \label{eq:travConst}
  B = c m_- - f_- = c m_+ - f_+
\end{equation}
From \eqn{travConst} we can also determine $c$
\begin{equation}
  \label{eq:travSpeed}
  c = \frac{f_+ - f_-}{m_+ - m_-}
\end{equation}
Note that \eqn{travSpeed} is the Rankine-Hugoniot condition.  We will
take $B=c m_+ - f_+$. The traveling wave equation is then
\begin{eqnarray}
\label{eq:travOde}
\phi(v)_\xi  &=& f(v) - \cbl \frac{f_+ - f_-}{m_+ - m_-} \sbl m(v) - m_+ \sbr + f_+\cbr \\
&=& h(v_-,v_+,v)
\end{eqnarray}
\Eqn{travOde}, with boundary conditions given by
\eqnst{travOdeData1}{travOdeData2}, may not have a solution, and, if
it does, solutions are not unique (if $u(\xi)$ is a solution, then
$u(C (\xi - \xi_0))$ is also a solution).  For two-phase flow and
Richards' equation and the variable transforms we have defined,
$\phi(u)$ is a monotone decreasing function of $u$.  Since we consider
the case $u_+ = 0$ and $0 < u_- <1$, this implies that $\phi(u_-) <
\phi(u_+)$.  The asymptotic boundary conditions can only be met if $h
> 0$ for $u_+ \leq u \leq u_-$.  The term in braces on the right-hand
side of \eqn{travOde} is an expression for the chord connecting $f_+$
and $f_-$. The requirement that $h>0$ can then be interpreted as a
constraint that the chord connecting $f_+$ and $f_-$ lie below the
graph of $f(m)$.  Rearranging the requirement that $h>0$ yields half
the Oleinik entropy inequality above. The other half comes from using
the second definition of the constant $B$ above.  Thus, the analysis
of the first-order case carries over to the full model in the
following way: traveling wave solutions exist whenever the solution of
the Riemann problem consists of a single shock wave. Dividing
\eqn{travOde} by its (positive) right hand side and integrating once
yields
\begin{equation}
  \label{eq:front}
  G(u) = \int^{u}_{u_0} \frac{\od{\phi}{u}}{f(u) - \sbl c(m(u) - m_+) - f_+\sbr} d u = \xi 
\end{equation}
where $u_0 = u(0)$ can be chosen arbitrarily in $(u_+,u_-)$. Thus, if
the integral in \eqn{front} is bounded as $u \rightarrow u_+$,
$u(\xi)$ goes to $u_+$ in ``finite $\xi$'' and the ``front''
$\zeta(t)$ is given by $tG(u_+)$.

Consider first the simplified model problem with $u_-=1$, $u_+=0$.
The integral above is
\begin{equation}
  \label{eq:simpModFront}
  \xi = \int^u_{u_0} \frac{q a u^{q-1}}{bu^p - c u} du
\end{equation}
If $b=1$, $p=2$ and $q=1$, the simplified model is viscous Burgers'
equation, and we have
\begin{equation}
  \label{eq:burgers}
  \xi(u) = \log(1-u) - log(u)
\end{equation}
Since $\xi(u)$ is unbounded as $u \rightarrow u_{\pm}$, we see that
there is no front ($\zeta(t)=\infty$). On the other hand, for $p=2$,
we have
\begin{equation}
  \label{eq:degenerateBurgers}
  \xi(u) = 2 \log(1-u)
\end{equation}
and therefore $\zeta(t) = 2t$. Note also that $\od{\xi}{u} = -2/(1-u)$
and therefore $-1/2 \leq \od{u}{\xi} \leq 0$: solutions are Lipschitz
continuous. For $p=3$ we have
\begin{equation}
  \label{eq:degenerateBurgers2}
  \xi(u) = 3(u + \log(1-u))
\end{equation}
so that $\zeta(t) = 3t$. However $\od{\xi}{u} = 3[1 - 1/(1-u)]$ and
therefore $\od{u}{\xi} \rightarrow -\infty$ as $x \rightarrow
\zeta(t)$: solutions fail to be Lipschitz continuous in space or time
at $\zeta(t)$.

\subsection{Barenblatt's Solution}

The traveling wave solutions don't exist in all of the situations
we're interested. We look at another set of solutions under the
assumption that advection is linear:
\begin{equation}
  \label{eq:porousMedia}
  u_t + (b u - au^p_x)_x = 0
\end{equation}
By changing variables to $\xi = x - bt$ and and seeking a solution of
the form $u(\xi,t) = t^{-\alpha} v(\xi t^{-\beta})$ it can be shown
that a solution on $(-\infty,\infty)$ is given by (c.f.
\citep{Evans_98})
\begin{equation}
\label{eq:barenblatt}
u(x,t) = \frac{1}{t^{\frac{1}{p+1}}}\sbl B - \frac{p-1}{2a p(p+1)} \pl \frac{x-bt}{t^{\frac{1}{p+1}}} \pr^2 \sbr^{\frac{1}{p-1}} \quad |x - bt| < \zeta(t)
\end{equation}
for $p \neq 1$, where $B$ is determined from the particular initial
conditions $u(x,0) = M \delta(x)$. For $p > 1$ the front location is
finite and given by
\begin{equation}
\label{eq:barenblattFront}
\zeta(t) = bt + \pl \frac{2ap(p+1)}{p-1} B \pr^{1/2} t^{\frac{1}{p+1}}
\end{equation}
For $p < 1$ there is no front. 

Taking a different perspective, if we follow a point moving with speed
b ($x - bt = x_0$) we see that it is driven toward equilibrium ($u =
0$) fastest for small $p$.
\section{Finite Element Formulations}

I'm going to try to write a fairly general discontinuous formulation
following \cite{Brezzi_etal_05}.  I'll start with a non-mixed
formulation. The classical (strong) statement of the problem is
\begin{eqnarray}
m_t + \deld \pl \vec f - \ten a \grad \phi \pr + r &=& 0 \quad \vec x \in \Omega\\
\phi &=& g \quad \vec x \in \Gamma_D \subset \partial \Omega \\ 
\pl \vec f - \ten a \grad \phi \pr \cdot \vec n &=& h \quad \vec x \in \Gamma_N \subset \partial \Omega
\end{eqnarray}
where $u$ is the unknown and everything else is possibly a nonlinear
function of $u$.  We assume that $\cbl \Omega_e \cbr$ is a partition
of the domain $\Omega$ into subdomains (elements) and $\cbl \Gamma_f
\cbr$ is the set of subdomain boundaries (faces) shared by two
distinct elements. We also set $I$ to be the set of indeces of
interior faces, $D$ the indeces for faces where Dirichlet conditions
hold on all nodes of the face and $N$ the indeces of faces where
Neumann conditions hold. We define $e_0(f),e_1(f)$ as the elements that
share thy internal face $f$ and $\vec n_{e_0(f)}$ as the outward normal
on $\partial \Omega_e$ along $\Gamma_f$. Likewise let $e(f), \vec
n_{e(f)}$ be the element and outward normal associated with faces on
the physical boundary. The solution and test space consists of
functions that are possibly discontinuous at the interior faces so we well reformulate the continuous problems for the partition $\cbl \Omega_e \cbr$. We define the following fairly standard notation for jumps and averages
\begin{eqnarray}
\dbl \phi \dbr &=& \cbl \begin{array}{ll} 
\phi_{e_0} \vec n_{e_0(f)} + \phi_{e_1(f)} \vec n_{e_1(f)} & f \in I \\
\phi_{e} \vec n_{e(f)} - g \vec n_{e(f)} & f \in D \\
\phi_{e} \vec n_{e(f)} & f \in N \end{array} \right. \\
\dbl \gvec \sigma \dbr &=& \cbl \begin{array}{ll} \gvec \sigma_{e_0(f)} \cdot \vec n_{e_0(f)} + \gvec \sigma_{e_1(f)} \cdot \vec n_{e_1(f)} & f \in I \\
\gvec \sigma_{e(f)} \cdot \vec n_{e(f)} - h & f \in N \\
\gvec \sigma_{e(f)} \cdot \vec n_{e(f)} & f \in D \end{array} \right. \\
\cbl \phi \cbr &=& \cbl \begin{array}{ll} 
\frac{\phi_{e_0(f)}  + \phi_{e_1(f)}}{2} & f \in I \\
\frac{\phi_{e(f)} + \phi(g)}{2} & f \in D \\
\phi_{e(f)} & f \in N \end{array} \right. \\
\cbl \gvec \sigma \cbr &=& \cbl \begin{array}{ll} 
\frac{\gvec \sigma_{e_0(f)} + \gvec \sigma_{e_1(f)}}{2} & f \in I \\
\frac{\gvec \sigma_{e(f)} + h \vec n_{e(f)}}{2} & f \in N \\
\gvec \sigma_{e(f)} & f \in D \end{array} \right. 
\end{eqnarray}
The continuous problem with reference to the partition is then formulated as
\begin{eqnarray}
m_t + r + \deld \pl \vec f - \ten a \grad \phi \pr &=& 0 \mbox{ on } \Omega_e \quad \forall e \\
\dbl \phi \dbr &=& 0 \mbox{ on } \Gamma_f \quad \forall f \in I \cup D \\
\dbl \vec f - \ten a \grad \phi \dbr  &=& 0 \mbox{ on } \Gamma_f \quad \forall f \in I \cup N
\end{eqnarray}
Let $\gvec \sigma =\vec f - \ten a \grad \phi $ and $\mathcal{R} = m_t
+ \deld \gvec \sigma + r$. We then define the following forms
\begin{eqnarray}
a_e(\mathcal{R},B_0 w;u) &=& \sum_e \int_{\Omega_e} \mathcal{R} B_0 w dV \\
a_{\phi}(\dbl \phi \dbr,\vec B_1 w;u) &=& \sum_{f\in D \cup I} \int_{\Gamma_f} \dbl \phi \dbr \cdot \vec B_1 w dS\\
a_{\sigma}(\dbl \gvec \sigma \dbr,B_2 w;u)  &=& \sum_{f \in N \cup I} \int_{\Gamma_f} \dbl \gvec \sigma \dbr B_2 w dS
\end{eqnarray}
where the third argument of the form is a reminder that they are nonlinear functions of $u$.
The weak formulation is then
\begin{equation}
a_e(\phi,B_0 w) + a_{\phi}(\dbl \phi \dbr, \vec B_1 w) + a_{\sigma}(\dbl \gvec \sigma \dbr, B_2 w) = 0 
\end{equation}
First I will define $B_0 w = w + \hat{B_0} w$ and put off thinking
about the definition of $\hat{B_0}$--this is where the residual based
stabilization term enters the formulation so there are many
alternatives. First I want to identify the portion of the weak
formulation that is calculated in the classical Galerkin method for
continous trial and test spaces. We can calculate $a_e(\mathcal{R},w)$ using
integration by parts to obtain
\begin{eqnarray}
a_e(\mathcal{R},w;u) &=& \sum_e \int_{\Omega_e} \sbl \pl m_t + r\pr w - \pl \vec f - \ten a \grad \phi \pr  \cdot \grad w \sbr dV + \sum_f \int_{\Gamma_f} \dbl \sigma w \dbr dS + \sum_{f \in N} \int_{\Gamma_f} hw dS  \\
&=& a_{CG}(\phi,w;u) + \sum_{f} \int_{\Gamma_f} \dbl \sigma \dbr \cbl w \cbr dS + \sum_{f} \int_{\Gamma_f} \cbl \sigma \cbr \cdot \dbl w \dbr dS +  \sum_{f \in N} \int_{\Gamma_f} hw dS  
\end{eqnarray} 
The Neumann conditions enters through the special defintions of $\dbl
\gvec \sigma \dbr$ on  $\Gamma_N$. Likewise, the second line follows from
the other definitions of jumps and averages and we define the form $a_{CG}(\phi,w;u)$. Now if we choose $B_2 w =
- \cbl w \cbr$, then $a_{\sigma}$ cancels with the second term in the
last line, except on $\Gamma_D$ so we obtain
\begin{equation}
  a_{CG}(\phi,w) + \sum_{f \in D } \int_{\Gamma_f} \dbl \sigma \dbr \cbl w \cbr dS + \sum_{f \in D \cup I} \int_{\Gamma_f} \cbl \sigma \cbr \cdot \dbl w \dbr dS + \sum_{f \in N} \int_{\Gamma_f} hw dS + a_{\phi}(\dbl \phi \dbr, \vec B_1 w) = 0 
\end{equation}
Now if the continuous functions are subspaces of the trial and test
space, we can approximate the solution by solving over this subspace.
But in this case the usual CG formulation results becuase
$a_{\phi} = 0$, $\dbl w \dbr = 0$, and $\cbl w \cbr|_{\Gamma^D} = 0$:
\begin{equation}
a_{CG}(\phi,w;u) + (h,w)_{\Gamma_N}
\end{equation}
or since the $\hat{B}_0$ term was not involved in any of the
calculations above we could have
\begin{equation}
  a_{CG}(\phi,w;u) + (h,w)_{\Gamma_N}  + a_e(\mathcal{R}, \hat{B}_0 w;u) = 0 
\end{equation}
A well-known problem with the CG formulation is that locally
conservative fluxes are not defined.  With the solution $\phi_{CG}$ in
hand, we therefore return to the original DG formulation under the
particular choice of $B_2$ (dropping the explicit nonlinearity in
$u$):
\begin{equation}
a_{CG}(\phi,w) + \sum_{f \in D } \int_{\Gamma_f} \dbl \sigma \dbr \cbl w \cbr dS + \sum_{f \in D \cup I} \int_{\Gamma_f} \cbl \sigma \cbr \cdot \dbl w \dbr dS + \sum_{f \in N} \int_{\Gamma_f} hw dS+ a_{\phi}(\dbl \phi \dbr, \vec B_1 w) = 0 
\end{equation}
A basis for the discontinuous functions can be generated by
multiplying the element basis functions by the element characteristic function $\xi_e$. We write the new basis as $\cbl
w_{e,i} = \xi_{e} \psi_{e,i} \cbr $. Now we solve the equation
above by {\em lagging} the CG terms and $\gvec \sigma $.
\begin{equation}
a_{CG}(\phi_{CG},w_{e,i}) + \sum_{f \in D } \int_{\Gamma_f} \dbl \sigma_{CG} \dbr \cbl w_{e,i} \cbr dS + \sum_{f \in D \cup I} \int_{\Gamma_f} \cbl \sigma_{CG} \cbr \cdot \dbl w_{e,i} \dbr dS + \sum_{f \in N} \int_{\Gamma_f} hw_{e,i} dS+ a_{\phi}(\dbl \phi_{DG} \dbr, \vec B_1 w_{e,i}) = 0 \label{conFlux}
\end{equation}
The only unknown term is then $\phi_{DG}$. Conveniently, all the other
terms can be computed by multiplying the CG residuals by $\xi_e$,
which are simply the element residuals that could have been saved from
the CG assembly process.  We can obtain a local method for
approximating $\phi_{DG}$ as follows. Fix global node $i$ and let
$\Omega_i = supp(\psi_i) \cap \cbl \Omega_e \cbr$. Then $\cbl \Omega_i
\cbr$ forms an overlapping decomposition of $\Omega$. Let
$\phi_{DG,i}$ be the solution of \ref{conFlux} over $\Omega_i$ with
boundary conditions $\phi_{DG,i} = 0$ on $\partial \Omega_i$. Now, if
we fix $\xi_e$ and sum these local equations over all $i$ we obtain
\begin{eqnarray}
\int_{\Omega_e} (m_t + r) dV &+& \sum_{f \in F(e) \cap D} \int_{\Gamma_f} \dbl \sigma_{CG} \dbr \cbl w_{e,i} \cbr dS + \sum_{f \in F(e) \cap (D \cup I)} \int_{\Gamma_f} \cbl \sigma_{CG} \cbr \cdot \vec n_e dS \\&+& \sum_{f \in F(e) \cap N} \int_{\Gamma_f} h dS+ \sum_{f \in F(e) \cap (D \cup I)} \int_{\Gamma_f} h_{DG} = 0
\end{eqnarray}
where 
\begin{equation}
h_{DG} = \sum_{i} \dbl \phi_{i} \dbr \cdot \vec B_1
\end{equation}
Summing over all $\Omega_e$ then yeilds
\begin{equation}
\int_{\Omega} (m_t + r) dV + \sum_{f \in D} \int_{\Gamma_f} \cbl \sigma_{CG} \cbr \cdot \vec n_e dS + \sum_{f \in N} \int_{\Gamma_f} h dS+ \sum_{f \in D} \int_{\Gamma_f} h_{DG} = 0
\end{equation}
Thus, the flux 
\begin{equation}
h_f = \cbl \sigma_{CG} \cbr + \sum_{i} \dbl \phi_{DG} \dbr|_{f} \cdot \vec B_1
\end{equation}
is the sought after locally and globally conservative flux for the
$CG$ solution. I'm not sure how to choose $\vec B_1 w$, but presumably
the literature on DG will provide guidance. I think $\vec B_1 w =
\frac{-1}{|\Gamma_f|} \dbl w \dbr$ may be good enough. It would be
interesting to then iterate over the entire process solving again
solving over a CG test space for the CG component $\phi_{CG}$.  In
this we we might achieve convergence to a DG solution while every
iterate would be ``locally and globally mass conservative''.

Now I'll return to the issue of stabilizing the $CG$ solution, or in
the notation above, the definition of $B_0$. I'll use the variational
multiscale framework for this purpose. We write the true solution to
the problem as $u = u_h + \hat{u} \in V$ where $u_h \in V_h$, $\hat{u}
\in \hat{V}$, and $V_h$ and $\hat{V}$ are linearly independent
subspaces of $V$. We similarly split the test space into linearly
independent subspaces $W_h$ and $\hat{W}$. This yields a coupled
system for $u_h,\hat{u}$
\begin{eqnarray}
 a_{CG}(\phi,w_h;u_h+\hat{u}) + (h,w_h)_{\Gamma_N}  + a_e(\mathcal{R}, \hat{B}_0 w_h;u_h+\hat{u}) &=& F_h \\ 
 a_{CG}(\phi,\hat{w};u_h+\hat{u}) + (h,\hat{w})_{\Gamma_N}  + a_e(\mathcal{R}, \hat{B}_0 \hat{w};u_h+\hat{u}) + &=& \hat{F} \\ 
\end{eqnarray} 
This is a nonlinear system of equations. If we write Newton's method for this system and split the corrections as $\delta u_h,\delta \hat{u}$ we obtain the linear 
\begin{eqnarray}
\int_{\Omega} \sbl \pl \od{m_t}{u} + \od{r}{u} \pr \delta u_h w_h - \pl \od{\vec f}{u} \delta u_h - \ten{a} \od{\phi}{u} \grad \delta u_h \pr \cdot \grad w_h \sbr dV + \\
\int_{\Omega} \delta \hat{u} \sbl \od{m_t}{u} w_h  + \od{r}{u} w_h - \od{\vec f}{u} \grad w_h + \deld \ten{a}^t \od{\phi}{u} \grad w_h \sbr dV + \mbox{BI's} = -F_h
\end{eqnarray}
and
\begin{eqnarray}
\int_{\Omega} \sbl \pl \od{m_t}{u} + \od{r}{u} \pr \delta \hat{u} \hat{w} - \pl \od{\vec f}{u} \delta \hat{u} - \ten{a} \od{\phi}{u} \grad \delta \hat{u} \pr \cdot \grad \hat{w} \sbr dV = \\
-\hat{F} - \sum_e \int \sbl \pl \od{m_t}{u} + \od{r}{u} \pr \delta u_h + \deld \pl \od{\vec f}{u} \delta u_h - \ten{a} \od{\phi}{u} \grad \delta u_h \pr \sbr \hat{w} dV + \mbox{BI's}
\end{eqnarray}
which we can write more compactly as 
\begin{eqnarray}
a(\delta u_h, w_h) + \sum_{e} (\delta \hat{u},\mathcal{L}^* w_h)_{\Omega_e} + \mbox{BI's} &=& - F_h \\
a(\delta \hat{u},\hat{w}) &=& - \hat{F} - \sum_e (\mathcal{L} \delta u_h - f,\hat{w}) + \mbox{BI's}
\end{eqnarray}
We now approximately solve for $\delta \hat{u}$ in terms of $\delta u_h$ and then solve the resulting equation for $\delta u_h$.
Taking continuous spaces for $u_h$ and $w_h$ and choosing $B_2$ as above yields
\begin{eqnarray}
a_e(\mathcal{R},B_0 w_h) + a_{\phi}(\dbl \phi \dbr, \vec B_1 w_h) + a_{\sigma}(\dbl \gvec \sigma \dbr, B_2 w_h) &=& 0 \\
a_e(\mathcal{R},B_0 \hat{w}) + a_{\phi}(\dbl \phi \dbr, \vec B_1 \hat{w}) + a_{\sigma}(\dbl \gvec \sigma \dbr, B_2 \hat{w}) &=& 0 
\end{eqnarray} 

We begin with the general ADR system
\begin{equation}
  \label{eq:nladrsysImp}
  \pd{m^i}{t} + \deld \sbl \vec f^i - \sum_{j=1}^{n_c} \ten{a}^{ij} \grad \phi^j \sbr + r^i = 0 \mbox{ in } (0,T] \times U \mbox{ for } i=1,\ldots, n_c
\end{equation}
which we write as an abstract evolution equation on some space $X$:
\begin{equation}
  \label{eq:abstractEvolve}
  \od{m^i}{t} = F^i \mbox{ for } i=1,\ldots, n_c
\end{equation}
where $m^i$ and $F^i$ are possibly nonlinear differential operators on
the solution $(u^1,\ldots,u^{n_c})$. This ``Rothe's'' Method for
solving time-dependent partial differential equations (a.k.a
discretizing in time first).  We consider linear multistep,
Runge-Kutta, and Rosenbrock methods for the discretization of this
problem.  Multistep methods take the form
\begin{equation}
  \label{eq:multistep}
  \sum_{j=1}^{n_s} a_j m(u_{n-j}) = \sum_{j=1}^{n_s} b_j F_{n-j}(t_{n-j},u_{n-j})
\end{equation}
where we have dropped the component superscript $i$ and interpret the
\eqn{multistep} as a system. Note that if $b_0 = 0$ then the method is
semi-explit for our problem. It would still remain to solve for $u$
based on the relation between $u$ and $m$, which is usually easy or
even completely trivial. To put the resulting equation into focus we write it in the simpler form
\begin{equation}
  \label{eq:multistepReduced}
  a_0 m_n - b_n F_n - \beta = 0
\end{equation}
The Runge-Kutta methods generally take the form
\begin{eqnarray}
  \label{eq:rungeKutta}
  m(u_{n+1}) &=& m_n + h \sum_{j=1}^{n_s} b_j F(t_n + c_j h,u_{nj}) \\
  m(u_{nk})  &=& m_n + h \sum_{j=1}^{n_s} a_{kj} F(t_n + c_j h,u_{nj}) 
\end{eqnarray}
Note that these equations represent an implicit system for
$u_{n+1},u_{n1},\ldots, u_{ns}$. If $a_{kj} = 0$ for $k>j$ then we can
solve this system sequentially so that that the solution can be
obtained in $n_s$ steps where a system of the same size as the
original PDE is solved at each stage. These are called diagonally
implicit methods (DIRK). Furtheremore if additionally $a_{kk} = 0$
then $m(u_{nk})$ is determined explicitly and only $m(u)$ need be
inverted to obtain $u_{nk}$. Such a method would correspond to explit
RK methods if $m=u$. Typically the implicit system in DIRKs is solved
using Newton's method. The idea behind Rosenbrock methods is to
approximate the intermediate stages using one Newton step. These
yields methods that are linearly implicit: for our problems the
require the solution of a linear PDE at each step rather than a
nonlinear PDE. If we write the RK method for a diagonally implicit
method as
\begin{eqnarray}
  \label{eq:rungeKutta}
  m(u_{n+1}) &=& m_n + h \sum_{j=1}^{n_s} b_j F(t_n + c_j h,u_{nj}) \\
  m(u_{nk})  &=& m_n + h \sum_{j=1}^{k} a_{kj} F(t_n + c_j h,u_{nj}) 
\end{eqnarray}
then the newton step for each stage takes the form

\bibliographystyle{plainnat} \bibliography{aw}
\end{document}
